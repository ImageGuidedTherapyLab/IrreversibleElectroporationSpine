\documentclass{article} 
%=================================================
%  possible font errors: fi fl
%=================================================
\usepackage{amssymb,amsfonts,amsmath}
\usepackage{color,graphicx}
\usepackage[left=1.0in,right=1.0in,top=1.0in,bottom=1.0in]{geometry}
\newcommand{\eqn}[1]{(\ref{#1})}
%http://latex2rtf.sourceforge.net/latex2rtf_1_9_19.html#Conditional-Parsing
%Starting with LaTeX2RTF 1.9.18, there is a handy method for
%controlling which content should be processed by LaTeX or by
%LaTeX2RTF . Control is achieved using the standard \if facility of
%TeX. If you include the following line in the preamble of your document 
%
%     \newif\iflatextortf
%Then you will create a new \iflatextortf command in LaTeX . TeX
%sets the value of this to false by default. Now, LaTeX2RTF
%internally sets \iflatextortf to be true, and to ensure that this
%is always the case, LaTeX2RTF ignores the command
%\latextortffalse. This means that you can control how different
%applications process your document by
%
%     \iflatextortf
%     This code is processed only by latex2rtf
%     \else
%     This code is processed only by latex
%     \fi
%Note that \iflatextortf will only work within a section; you
%cannot use this command to conditionally parse code that crosses
%section boundaries. Also, it will only work on complete table or
%figure environments. Due to the mechanism used by LaTeX2RTF in
%processing these environments, at this time the only way to
%conditionally parse tables and figures is to include two complete
%versions of the environment in question, nested within an
%appropriate \iflatextortf structure.
%
\newif\iflatextortf

\iflatextortf 
%do nothing
\else  %pdflatex
\usepackage{boxedminipage,float}
\usepackage{wrapfig,setspace}
\newcommand{\picdir}{pdffig}
\fi

\usepackage[pdftex, plainpages=false, colorlinks=true, citecolor=black, filecolor=black, linkcolor=black, urlcolor=black]{hyperref}

%=================================================
\begin{document}

\title{\bf \Large
IRE Methods
}

\author{ D.~Fuentes$^1$
}

%\date{ \small
%$^1$The University of Texas M.D. Anderson Cancer Center,\\
%Department of Imaging Physics, Houston TX 77030, USA \\
%$^2$BioTex, Inc., Houston TX 77054, USA\\
%Email: \texttt{fuentesdt@gmail.com, jstafford@mdanderson.org}   \\
%% Webpage: \texttt{http://wiki.ices.utexas.edu/dddas}
%}

%\date{Received: April 2010 / Accepted: XXX }
% The correct dates will be entered by the editor


\maketitle

\paragraph{Keywords} Bioheat Transfer $\cdot$ 
                     %MR Thermal Imaging $\cdot$ 
                     Laser Tissue Interaction $\cdot$ 
                     Finite Element Modeling

%===================================================================
\section{Methods}
%===================================================================

%The 32 slices that made up the 3D image were loaded into ITK-SNAP (Penn
%Image Computing and Science Laboratory (PICSL), Dept. of Radiology,
%University of Pennsylvania) an application for segmenting 3D images.
%Once generated from the 3D image the segmented brain data was saved and
%loaded into Cubit (Sandia National Laboratories Albuquerque, New Mexico)
%to be meshed.  Cubit generated a mesh of hexahedral elements throughout
%the 3D volume of the brain. 

%Following the creation of the brain mesh a second mesh object
%representing the laser source was constructed and placed at the
%appropriate location within the simulated brain. 
The thermal effects are expected to be confined to a region near
the applicator. Consequently, the boundary conditions of a
conformal canine specific 3D volumetric hexahedral FEM mesh, shown
in Figure~\ref{canineSetup} are expected to have no effect on the
results.  Thus, a single template FEM mesh was used to investigate
each canine data set. Magnitude images were used to manually
register the mesh template to the thermal imaging data.  The
objective of this study was to investigate modeling error in an ROI
about the active tip of the applicator.  

% full mesh resolution
%          elements    nodes
%mesh[0]     20069     22148 
%mesh[1]     27041     29568
%mesh[2]     47785     51366
%mesh[3]     52232     56565

% summary of mesh resolution across ROI
%                   meshlowerres    meshlores    meshnormres     mesh hires       original 
%(element,nodes)      mesh[0]        mesh[1]        mesh[2]        mesh[3]        mesh[4]    
%     ROI[1]      ( 2002, 2580)  ( 2678, 3388)  ( 8127, 9394)  ( 4200, 5200)  (10603,12324)
%     ROI[2]      ( 1513, 2076)  ( 1975, 2677)  ( 6665, 7953)  ( 3341, 4227)  ( 8091, 9741)
%     ROI[3]      ( 2393, 3029)  ( 3131, 3951)  ( 9013,10352)  ( 4896, 5974)  (11978,13874)
%     ROI[4]      (  896, 1349)  ( 1225, 1806)  ( 4045, 5120)  ( 2629, 3481)  ( 5523, 6989)
%
%original ROI data for mesh[0]
% ROI 1: # elements 10603  # nodes 12324  FOV 2.0cm x 2.1cm x 1.5cm
% ROI 2: # elements 8091   # nodes 9741   FOV 1.6cm x 2.5cm x 1.3cm
% ROI 3: # elements 11978  # nodes 13874  FOV 2.1cm x 2.6cm x 1.6cm
% ROI 4: # elements 5523   # nodes 6989   FOV 2.1cm x 2.3cm x 0.8cm
%elems = [10603,8091,11978,5523]  
%nodes = [12324,9741,13874,6989]  
%elems = [ 2002,2678,8127,4200,10603, 1513,1975,6665,3341, 8091, 2393,3131,9013,4896,11978, 896,1225,4045,2629, 5523 ]
%nodes = [2580, 3388, 9394, 5200,12324, 2076, 2677, 7953, 4227, 9741 , 3029 , 3951,10352, 5974,13874 , 1349, 1806 , 5120, 3481, 6989]
%
%xroi  = [2.0  ,1.6 , 2.1 , 2.1]  
%yroi  = [2.1  ,2.5 , 2.6 , 2.3]  
%print mean(elems) , std (elems)
%print mean(nodes) , std (nodes)
%print mean(xroi ) , std (xroi )
%print mean(yroi ) , std (yroi )
%>>> print mean(elems) , std (elems)
%9048.75 2467.00206475
%>>> print mean(nodes) , std (nodes)
%10732.0 2617.19114701
%>>> print mean(xroi ) , std (xroi )
%1.95 0.206155281281
%>>> print mean(yroi ) , std (yroi )
%2.375 0.192028643697


%>>> from numpy import mean
%>>> from numpy import std
%>>> elems = [10603,8091,11978,5523]
%>>> print mean(elems) , std (elems)
%9048.75 2467.00206475
%>>> elems = [ 2002,2678,8127,4200,10603, 1513,1975,6665,3341, 8091, 2393,3131,9013,4896,11978, 896,1225,4045,2629, 5523 ]
%>>> print mean(elems) , std (elems)
%4746.2 3196.00870775
%>>> nodes = [2580, 3388, 9394, 5200,12324, 2076, 2677, 7953, 4227, 9741 , 3029 , 3951,10352, 5974,13874 , 1349, 1806 , 5120, 3481, 6989]
%>>> print mean(nodes) , std (nodes)
%5774.25 3593.95211258
%>>> print min(nodes) , median (nodes) , max (nodes)
%1349 4673.5 13874
%>>> print min(elems) , median (elems) , max (elems)
%896 3693.0 11978


\textcolor{blue}{ % comments for review 1
The ROI for each canine, $\Omega \subset \mathbb{R}^3$ (FOV=
1.95x2.38cm $\pm$ .21x.19cm) 
was chosen large enough to encompass the
extent of the heating region but small enough to minimize bias of
the error metric \eqn{weightedL2Norm} where no  heating occurred.
Individual regions of the brain were not segmented due to the
localized nature of the heating study.
To ensure convergence, five FEM mesh resolutions were considered for
each ROI (min/median/max \# elements = 896/3693/11978, 
          min/median/max \# nodes    = 1349/4674/13874).
Within the ROI, the mesh size element diameter ranged between
[0.5mm,2mm]. As a reference, 1mm is on the order of the pixel size.
}


Geometric details of the applicator were closely modeled according
to the manufacturer design and details provided in
Figure~\ref{canineSetup}.  The active cooling of the applicator was
\textcolor{blue}{ % comments for review 1
modeled using Dirichlet boundary conditions to fix the temperature
}
along the axial length of the 1.5mm diameter catheter.  The
temperature of the catheter was studied at both ambient room
temperature, 21$^o$C, and body temperature 34.6$\pm$ 0.7 $^o$C.
The 400 $\mu$m core diameter active region of the fiber-optic was
modeled as $\Omega_{tip}$, a cylindrical 1.5mm diameter, 1cm axial
region, at a distance of 5mm from the catheter tip. Modeling at the
level of the 400 $\mu$m core diameter active tip is not expected to
produce significant differences in the computations.  The finite
element basis functions are continuous across the applicator tissue
interface.  

The optical photon source of the laser-tissue interaction was
rigorously derived from a diffusion theory approximation to the
light transport equation~\cite{Welch95}.  The resulting photon
source is referred to as the conformal discretization approximation
(CDA) in this manuscript.  The CDA model is similar to fully
analytic attempts to capture the cylindrical geometry of the
interstitial fiber-optic~\cite{Dickey04} but exploits the
underlying finite element discretization of the computational
domain to evenly distribute the laser power amongst the axial
length and finite diameter in the active tip region.  An analogous
discretization approach was implemented on structured
grids~\cite{arnfield1989optical} and results in multiple individual
optical diffusion approximation (ODA) along the centerline of the
applicator.  The CDA accounts for the finite diameter of the
applicator.  Within the diffusion theory assumptions, the resulting
conformal discretization approximation is expected to provide a
fluence source comparable to direct numerical solutions to the
light transport equation~\cite{Mohammed05,shafirstein2004}.

The diffusion model for light transport is built from the
assumption that light is scattered more than absorbed, $ \mu_a <<
\mu_s 
% \Rightarrow \mu_{eff} << \mu_t
$, where $\mu_a$ and $\mu_s $ give probability of absorption and
scattering of photons, respectively.  An applied volumetric power
density, $P^* \left[\frac{W}{m^3}\right]$, provides a photon flux
throughout the active region of the interstitial laser fiber,
$\Omega_{tip}$.  Huygens superposition principle~\cite{Dickey04} is
used to treat each position in the domain, $\hat{x}\in
\Omega_{tip}$, as an isotropic  differential point source of
irradiance, $dE$,
\[
dE(x,\hat{x})= 
             \frac{P^*(t)d\hat{x}}{4\pi 
    \| x - \hat{ x}\|^2}
             \exp\left(\mu_t
                         \| x - \hat{ x }\|\right) 
\qquad \qquad
\hat{\textbf{s}} = 
\frac{ x - \hat{ x }}{ \| x - \hat{ x}\| }
\qquad \qquad
x \in \Omega \backslash \Omega_{tip}
\]
The total attenuation is denoted $\mu_t  = {\mu_a} + {\mu_s}$.  The
propagation direction, $\hat{\textbf{s}}$, is the unit vector from
the primary source of unattenuated photons to the position $x$.
The light transport is assumed quasi-static and a stationary
diffusion model is used at each time point of interest to relate
the scattered fluence, $z$, to the known irradiance, $dE$, emitted
from point source $\hat{x} \in \Omega_{tip}$
%
% a unit vector along the line connecting the source (or element of the
% source) to the point x"
%
% "E_0(x,s) is the irradiance at point x in the absence of tissue and s
% indicates the direction of propagation of primary light. That is, s is
% a unit vector along the line connecting the source (or element of the
% source) to the point x"
%
% Use Welch...
% The radiance of the primary light written in terms of its irradiance 
% in equation (6.22). Substitutive this into eqn (6.16) leads to 
% the transport equation written in terms of the scattering light
% as the unknown variable with the unscattered light as the source,
% eqn (6.26). IE eqn (6.16) and (6.26) are EQUIVALENT given eqn (6.16)
% Under these assumptions 
%
% diffusion approximation introduced as a truncated series expansion of
% the radiance of scattered light eqn (6.27) and is used in the
% derivation of the flux conservation, eqn (6.32). The energy
% conservation, eqn (6.29), is EXACT.
%
% F in eqn (6.29) is  the net energy flux of the SCATTERED LIGHT 
% NOT THE SCATTERED PLUS IRRADIANT see eqn (6.28a) in Welch
%
% substituting  eqn (6.32b) into eqn (6.29) of welch book
\begin{equation} \label{odaLTE}
% \frac{1}{c(x)}
% \frac{\partial(z+E)}{\partial t}
 -{\mu_a} z 
 +{\mu_s} \; dE(x,\hat{x})
 = \nabla \cdot 
   \left( 
   - \frac{ \nabla z }{3\mu_{tr}} 
   + \frac{ {\mu_s} g }{3\mu_{tr}}
      \; dE(x,\hat{x}) \; \hat{\textbf{s}}
   \right)
\quad
  \mu_{tr}  = 
                 {\mu_a} + 
                    {\mu_s} (1-g)
\quad
x \in \Omega \backslash \Omega_{tip}
\end{equation}
The anisotropic factor is denoted $g$. 
A conformal discretization approximation (CDA)
of the integral of the differential irradiance over
the domain of the active tip, $\Omega_{tip}$,
is used to obtain an analytical expression
for the scattered light, $z$
\[
\begin{split}
E  & = \int_{\Omega_{tip}} dE(x,\hat{x}) d\hat{x}
     = \int_{\Omega_{tip}} 
             \frac{P^*(t)}{4\pi \| x - \hat{ x}\|^2}
             \exp\left(\mu_t
                         \| x - \hat{ x}\|\right) d\hat{x}
\\
  & \approx 
  \sum_{e\in \Omega_{tip}} E_e(x,t)
  =
  \sum_{e\in \Omega_{tip}} 
          \frac{\Delta V_e P^*(t)}{4\pi \| x - \hat{ x}_e\|^2}
          \exp\left(\mu_t 
                 \| x - \hat{ x}_e\|\right) 
\end{split} 
\]
where $\Delta V_e$ is the volume and $\hat{ x}_e$ is the
centroid of the elements within the FEM mesh of the active tip. 
Here, conformal refers to a discretization such that the finite
elements adhere to the boundary of the active tip, $\Omega_{tip}$.
Similarly for the flux,
%FIXME  FIXME  FIXME  FIXME  FIXME  FIXME  FIXME  FIXME  FIXME  FIXME
%Should ${\mu_t}$ be the average over the path length???
% this is already an approximation... if go through the trouble to 
% take \mu_t as the average, what will be gained... ie don't worry about
% it
%FIXME  FIXME  FIXME  FIXME  FIXME  FIXME  FIXME  FIXME  FIXME  FIXME
\[
\begin{split}
\int_{\Omega_{tip}} dE(x,\hat{x}) \; \hat{\textbf{s}} \; d\hat{x}
   &   = \int_{\Omega_{tip}} 
             \frac{P^*(t)}{4\pi \| x - \hat{ x}\|^2}
             \exp\left(\mu_t 
                         \| x - \hat{ x}\|\right) 
\frac{ x - \hat{ x} }{ \| x - \hat{ x}\| }
d\hat{x}
 \\
  & \approx 
    \sum_{e\in \Omega_{tip}} E_e(x,t) \hat{\textbf{s}}_e
  = \sum_{e\in \Omega_{tip}} 
          \frac{\Delta V_e P^*(t)}{4\pi \| x - \hat{ x}_e\|^2}
          \exp\left(\mu_t 
                         \| x - \hat{ x}_e\|\right) 
\frac{ x - \hat{ x}_e }{ \| x - \hat{ x}_e\| }
\end{split} 
\]
By linearity, each element in the discretization may be treated as
an uncoupled and independent source in the light diffusion
equation.
\[
 -{\mu_a} z_e 
 +{\mu_s} E_e
 = \nabla \cdot 
   \left( 
   - \frac{ \nabla z_e }{3\mu_{tr}} 
   + \frac{ {\mu_s} g }{3\mu_{tr}} 
       E_e 
      \frac{ x - \hat{ x}_e }{ \| x - \hat{ x}_e\| }
   \right) \qquad \forall e \in \Omega_{tip}
\]
The total fluence resulting from each element source, $(\phi_t)_e =
z_e + E_e$, may be obtained from the classical isotropic point
source solution~\cite{Welch95} as  the sum of the light scattered
from the element, $z_e$, and the element-wise primary source,
$E_e$.  The total emanating fluence, $\phi_t(x,t)$, is the
superposition of the element-wise solutions and reduces to a volume
weighted sum over the elements.
\begin{equation} \label{wfsLaserFluence}
\begin{split}
   \phi_t(x,t) & = \sum_{e \in \Omega_{tip}}
 {P^*(t)} \Delta  V_e  \left(
   \frac{3\mu_{tr} \exp(-\mu_{eff} \| x -{ x_e}\|) }
      {4\pi \| x-{ x_e}\|}
 - 
      \frac{ 
             \exp(-\mu_t \| x -{ x_e}\|) }
           {2\pi \| x-{ x_e}\|^2}
   \right)
\\
& \approx
    \sum_{e \in \Omega_{tip}}
 {P^*(t)}  \Delta V_e 
   \frac{3\mu_{tr} \exp(-\mu_{eff} \| x -{ x_e}\|) }
      {4\pi \| x-{ x_e}\|}
\qquad
 \mu_{eff}  = 
           \sqrt{ 3 {\mu_a} \mu_{tr} }
\\
& 
\hspace{3in}
 x \in \Omega \backslash \Omega_{tip}
\end{split}
\end{equation}

A single point source ODA of the photon source at the centroid of
the active tip, $ x_0$, is used as a control and comparison against
previous work~\cite{fuentesetal09}.
\begin{equation} \label{odaLaserFluence}
   \phi_t(x,t)  = 
   P(t) 
   \frac{3\mu_{tr} \exp(-\mu_{eff} \| x -{ x_0}\|) }
      {4\pi \| x-{ x_0}\|}
\qquad  x \in \Omega \backslash \Omega_{tip}
\end{equation}

A linear Pennes model was used to simulate the bioheat
transfer~\cite{Pennes48}.  The fluence, $\phi_t$, provides the
thermal source in the bioheat equation and couples the diffusion
theory model of light transport in tissue to Pennes model.  The
laser powers and exposure times for each canine shown in
Figure~\ref{data980CanineSummary} were modeled using piecewise
continuous step functions of the wattage.  The model considers the
optical and thermal properties of the tissue as well as a highly
simplified model of the micro vasculature heat sink consisting of a
linear temperature difference between the heated region and the
arterial temperature, $u_a$. The Pennes bioheat equation has been
shown to accurately model heating in areas absent of major
vasculature~\cite{arkin1994}. 
\[ \begin{split}
 \rho  c_p \frac{\partial u}{\partial t}
 -\nabla \cdot ( {k} \nabla u) 
 +{\omega} c_{blood} &(u - u_a )
 = {\mu_a} \phi_t  \qquad \text{in } \Omega
\\
   -  k  \nabla u \cdot \textbf{n} = 0
           \qquad \text{on } \partial \Omega
  &\qquad \qquad 
   u  = u_{probe} \qquad \text{in } \Omega_{probe}
\end{split} 
\]
The initial temperature field, $u( x,0)=u^0=$ 34.6$\pm$ 0.7 $^o$C,
is taken as the measured baseline body temperature.  The density of
the continuum is denoted $\rho$ and the specific heat of blood is
denoted $c_{b} \left[\frac{J}{kg \cdot K}\right]$.  The boundary of
the mesh template is far enough away from the heating region such
that zero heat flux may be considered as the Neumann boundary
condition on $\partial \Omega$.  The scalar-valued coefficient of
thermal conductivity and perfusion are denoted $k$ and $\omega$.
Within the region of the mesh that represents the applicator,
$\Omega_{probe}$, a Dirichlet boundary condition is imposed such
that the cooling water of the applicator is held at body
temperature, $u_{probe}=u^0$, or ambient temperature,
$u_{probe}=$21$^o$C, throughout the procedure.  The temperature
evolution predicted by the bioheat equation was computed using the
finite element method with linear polynomials and a Crank-Nicholson
time stepping scheme.

%%%%%%%%%%%%%%%%%%%%%%%%%%%%%%%%%%%%%%%%%%%%%%
\subsubsection{Constitutive Data}
%%%%%%%%%%%%%%%%%%%%%%%%%%%%%%%%%%%%%%%%%%%%%%

The bio-thermal and optical parameters were obtained from
literature~\cite{Handbook05,Welch95,duck1990} and were modeled as
homogeneous throughout the delivery region of interest,
Table~\ref{modeldata}.  An average value of the range of
scattering, $\mu_s$, and absorption, $\mu_a$, values quoted in the
literature was used.  A range of perfusion values
$\omega$=0.0,3.0,6.0,12.0$[\frac{kg}{s m^3}]$ were simulated to
capture physically realistic values of brain tissue perfusion found
in literature and as a first order study of temperature dependent
perfusion.  Empirical models of the temperature dependence of the
constitutive data~\cite{pegau1997absorption,Duggan00,Handbook05},
$k(u)$, $\omega(u)$, $\mu_a(u)$, will be considered in future
validation studies.  Statistics were collected comparing the
thermal imaging data for each animal against each permutation of
laser model, cooling model, and perfusion value, a total of 64
simulations at each mesh resolution considered. 

\begin{table}[h]
\caption{Constitutive Data~\cite{Handbook05,Welch95,duck1990}}\label{modeldata}
\centering 
\begin{tabular}{|c|c|c|c|c|c|c|c|} \hline 
$k $ $ \frac{W}{ m \cdot K}$ & $\omega$ $\frac{kg}{m^3 s}$ &  $g$  &  $\mu_s$ $\frac{1}{cm}$  &  $\mu_a$  $\frac{1}{cm}$   &  $\rho$ $\frac{kg}{m^3}$ &   $c_{blood}$ $ \frac{J}{kg \cdot K}$ &  $c_p$ $\frac{J}{kg \cdot K}$ \\ \hline
          0.527              &     0.0,3.0,6.0,12.0        & 0.862 &        47.0              &       0.45                 &  1045                        &            3840                      &                  3600          \\ \hline
\end{tabular}
\end{table}


%%%%%%%%%%%%%%%%%%%%%%%%%%%%%%%%%%%%%%%%%%%%%%%%%%%%%%%%%%%%%%%%%%%
\subsubsection{Theoretical Probability Density}
%%%%%%%%%%%%%%%%%%%%%%%%%%%%%%%%%%%%%%%%%%%%%%%%%%%%%%%%%%%%%%%%%%%
The theoretical probability density, $\Theta$, represents the
mathematical model predicted relationship between the parameter
space, $\mathbb{M}$, and the data space, $\mathbb{D}$. 
For a well defined mathematical model in which a unique solution
exists, the theoretical probability density may be taken as a delta
functional.
This assumption explicitly ignores any potential modeling error
and enforces the notion that the event of a temperature field not
predicted by the theory has zero probability of occurring.
The nonlinear mapping from parameter space to data space, 
$u(\vec{m}) \in \mathbb{D}$, 
is obtained as a solution to the Pennes bioheat
transfer model~\cite{Pennes1948} coupled with the
diffusion theory of light interaction with tissue~\cite{Welch95}. 
\[ 
 \rho  c_p \frac{\partial u}{\partial t}
 -\nabla \cdot ( {k(\textbf{x})} \nabla u) 
 + {\omega(\textbf{x})} c_{blood} (u - u_a )
 = {\mu_a(\textbf{x})} \varphi(m,\textbf{x},t)  \quad \text{in } \Omega
 \]
\[
   - {k(\textbf{x})} \nabla u \cdot \textbf{n} = 0
        \qquad \text{on } \partial \Omega
\qquad
   u(\textbf{x},0) = u^0 \qquad \text{in } \Omega
\]
The initial temperature field, $u(\textbf{x},0)=u^0$, 
is taken as the measured baseline body
temperature.  The density of the continuum is denoted $\rho$, the
specific heat of blood is denoted $c_{b} \left[\frac{J}{kg \cdot
K}\right]$, $k(\textbf{x})$ denotes the thermal conductivity, and
$\omega(\textbf{x})$ denotes perfusion.
Zero heat flux Neumann data is assumed on the boundary, $\partial \Omega$.
Although likely to result in optical parameter estimation 
error due to the strong scattering assumptions~\cite{carp2004radiative},
the classical
spherically symmetric isotropic standard diffusion approximation (SDA)
to the transport equation of
light within a laser-irradiated tissue~\cite{Welch95} is used as the kernel
in modeling the laser source term.  
The SDA was used in this
feasibility study for its computational simplicity in managing
the algorithmic complexity of computing the gradient and Hessian of the misfit function, 
Section~\ref{GradientHessianEvaluation}.
The optical-thermal response to the laser source,
$q_{laser} = {\mu_a(\textbf{x})} \varphi(m,\textbf{x},t)$,
is modeled as 
\[
\begin{split}
   \varphi(m,\textbf{x},t) & = 
    \int_{U_{tip}}
 \frac{P(t)  }{\text{Vol}(U_{tip}) }  
   \frac{3\mu_{tr} \exp(-\mu_{eff} \| \textbf{x} -\mathbf{ \xi}\|) }
      {4\pi \| \textbf{x}-\mathbf{ \xi}\|} \; d\mathbf{\xi}
\quad
\\
 & \hspace{2in}   x \in U \backslash U_{tip}
\\
 & 
  \mu_{tr}   = {\mu_a(\textbf{x}) } +  
               {\mu_s(\textbf{x})} (1-g)
\qquad
  \mu_{eff}  = \sqrt{3 {\mu_a(\textbf{x})} \mu_{tr}}
\end{split}
\]
$P(t) $ is the
laser power as a function of time, $\mu_a$ and $\mu_s $ are laser
coefficients related to laser wavelength and give probability of
absorption and scattering of photons, respectively. The anisotropic
factor is denoted $g$  
and $\|x-\xi\|$ denotes the distance from the laser photon source, $U_{tip}$.
Active cooling of the applicator is modeled by holding the temperature
in this region fixed through Dirichlet boundary conditions at the ambient temperature
\cite{fuentesetal11a}.

Given an initial temperature field, $u^0$, a set of model parameters,
$
m \equiv ( k(\textbf{x}), \omega(\textbf{x}), 
           \mu_a(\textbf{x}),\mu_s(\textbf{x})) \in \mathcal{M}
$, and the appropriate boundary condition, a model prediction of
the temperature field, $u(x,t)$, may be obtained.
Discretizing the temperature with a Galerkin expansion at each 
time step, the data space is represented by
the vector space of Galerkin
coefficients for the thermal field.
Similarly, a Galerkin discretization is  used for the model
parameters. The model space, $\mathbb{M}$, is the space of model
parameters for which a unique solution is well defined.
\[
\begin{split}
u(x,t^k) = \sum_j^{N_{dof}}  d_j^k  \phi_j(x) 
\qquad 
k = 1, 2, ..  N_{step}
\qquad
\qquad
\mathbb{D} = \mathbb{R}^{N_{dof}*N_{step}}
\\
m(x) = \sum_j^{N_{model}}  m_j  \psi_j(x) 
\qquad
\qquad
\mathbb{M} \subset \mathbb{R}^{N_{model}}
\end{split}
\]
Here $N_{dof}$ and $N_{model}$ are the spatial temperature degrees of freedom and the parameter
degrees of freedom, respectively. The number of time steps in the discretization is denoted $N_{step}$.
In this application, an element of the data space, $\vec{d} \in
\mathbb{D}$, is a set of Galerkin coefficients that may represent
either an experimentally observed MRTI quantity or a quantity
predicted by the bioheat transfer equation.  

%%%%%%%%%%%%%%%%%%%%%%%%%%%%%%%%%%%%%%%%%%%%%%
\subsection{Discretization and Constitutive Data}~\label{DiscretizationConstitutive}
%%%%%%%%%%%%%%%%%%%%%%%%%%%%%%%%%%%%%%%%%%%%%%

The finite element model of the \textit{in vivo} TVT inoculated canine
MRgLITT is shown in Figure~\ref{nanodog}. The insertion of the
applicator into the tissue was modeled using a mesh that consisted of
distinct regions for the applicator, healthy tissue, and cancer burden.
A quadrilateral mesh was extruded axially along the applicator to create
the base of the hexahedral finite element mesh. The mesh for the tissue
conforms to the boundary of the application and extends to the boundary
of the brain to ensure that the boundary does not influence the heating.
The entire mesh consists of $N_{dof}=$47,941 total nodes (grid points).
At each time step of the Crank-Nicolson scheme used, the temperature
across the applicator was held constant to model the effect of the
room-temperature cooling fluid which protects the laser fiber during
heating. The degrees of freedom across the applicator were held at
21$^o$C by treating the corresponding degrees of freedom as Dirichlet
boundary data. The bioheat transfer model temperature prediction is not
seen to be very sensitive to the optical scattering~\cite{Fengetal07}. 
Only optical absorption and thermal conductivity
parameter estimation is considered in this work.
The combined thermal conductivity and optical
absorption estimation consisted of $N_{model}=$4,570
spatially varying parameters. 
The remaining input parameters are assumed deterministic.
 The laser power profile used
during the therapy is shown in Figure~\ref{PowerProfile}. The test
pulse, 3W for 30 seconds, and main therapeutic pulse, 3.5W for 3min, are
shown. 
%There are many combinations active power durations and magnitude,
%including the data available, that may provide the significant heating
%and fluence needed to recover the optical parameters and subsequently
%predict the therapeutic heating accurately. 
The amount, duration, and magnitude of heating data needed to accurately
recover the optical parameters during the ``test" pulse of the
experimental data and subsequent critical evaluation of accuracy of the
calibrated computer model prediction during the therapeutic delivery is
considered outside the scope of this work. The current work seeks to
demonstrate the potential of parameter estimation, hence, parameter
estimation for verifying uptake and potentially planning the procedure
is considered for the time interval of the strongest fluence signal for
the data acquired. Data acquired during the time window of the
therapeutic heating and subsequent cooling, $\delta t = [204,504]$,
$N_{step} = 300$ seen
in Figure~\ref{PowerProfile}, is used to demonstrate the feasibility of
recovering the 3D spatially varying optical distribution.  The
Crank-Nicolson scheme was used to propagate the temperature field at one
second time steps.  Table~\ref{modeldata} summarizes the constitutive
data used.

\begin{table}[h]
\caption{Constitutive Data~\cite{Handbook05,Welch95,duck1990}}\label{modeldata}
\centering 
\begin{tabular}{|c|c|c|c|c|c|c|c|} \hline 
$[k^{lb},k^{ub}]$ $ \frac{W}{ m \cdot K}$ & $\omega$ $\frac{kg}{m^3 s}$ &  $g$  &  $\mu_s$ $\frac{1}{m}$  &  $[\mu_a^{lb},\mu_a^{ub}]$  $\frac{1}{m}$   &  $\rho$ $\frac{kg}{m^3}$ &   $c_{blood}$ $ \frac{J}{kg \cdot K}$ &  $c_p$ $\frac{J}{kg \cdot K}$ \\ \hline
          [0.1, 0.7]                    &             9.0             & 0.88  &       34.0e3              &     [60.0, 600.0]                             &  1045                        &            3840                      &                  3600          \\ \hline
\end{tabular}
\end{table}

The field was optimized in a 1.5cm diameter spherical ROI about
the laser applicator centroid.


%========================================================================
\subsection{Computational Methods}
%========================================================================

The three dimensional mathematical models for RF ablation are based on the
Pennes bioheat transfer model~\cite{Pennes48} coupled with the quasistatic
electrical conduction of RF waves as the primary mechanism of energy
transfer~\cite{chang2004tml}. The quasistatic equation for the voltage, is
obtained from the Maxwell equations with an irrotational electric field. The
RF energy delivered over the time interval of the treatment, $[0,T]$,
provides a resistive heating due to the current flow near the electrode of a
480kHz monopolar RF generator.  The perfusion coefficient, thermal
conductivity, and electrical conductivity, are assumed linear and spatially
homogeneous model parameters.  The computational domain is taken large
enough so that there may be assumed no heat flux across the surface.  The
boundary is assumed an electrical ground for the voltage.  The computational
models used provides a 4D prediction of both the temperature, $u(x,t)$, and
the voltage, $\phi(x,t)$.  The set of governing partial differential
equations that embody the computational models and as well as a summary of
the model parameters used is provided in the appendix.

The temperature of the blood warmed by a heat exchanger was used
as the initial temperature field, $u(x,0)=u^0$=37$^o$C.
Tissue damage is characterized by an Arrhenius model 
of the spatially varying damage field $D(x)$ resulting from the
time-temperature history during the RF ablation procedure.
\[
%\scalebox{1.0}{\includegraphics*{\picdir/arrheniusModel}}  
D(\textbf{x})  = 
    \int_0^T A \; e^{\frac{-E_a}{R \; u}} \; dt 
\]
The Arrhenius model is a classical and empirical law used to model
kinetics of chemical reaction adopted to describe tissue
damage~\cite{HM5,HM1,HM2}. Although there are more accurate damage models
available~\cite{fengetal08}, Arrhenius model is
commonly used due to its simplicity. In the Arrhenius model, the
parameters $E_a$, $A$, and $R$ are constants determined usually by {\em
in vitro} experiments, which represent activation energy, frequency
factor, and the universal gas constant, respectively. 
The values $A$ = 3.1e98$\frac{1}{s}$, $E_a$=6.28e5$\frac{J}{mol}$,
and $R$=8.314$\frac{J}{kg \; mol}$ were used in this study.

%========================================================================
\paragraph{\textbf{{\large Appendix}}}
%========================================================================

The equations governing the resistive heating that provides the thermal
source for the bioheat equation~\eqn{bioheatmodel} may be obtained from
Maxwell's equation~\cite{demkowicz2006cha} using Faraday's
law~\eqn{FaradayLaw}, Ampere's law~\eqn{AmpereLaw}, and the continuity
of free charge~\eqn{ChargeContinuity}.
\begin{equation} \label{FaradayLaw}
%  \scalebox{1.0}{\includegraphics*{\picdir/FaradayLaw}}  
 \nabla \times \vec{E} = -\frac{\partial}{\partial t} \vec{B}
\end{equation}
\begin{equation} \label{AmpereLaw}
%  \scalebox{1.0}{\includegraphics*{\picdir/AmpereLaw}}  
 \nabla \times \vec{B} = 
        \mu \vec{J} + \mu \epsilon \frac{\partial}{\partial t} \vec{E}
\end{equation}
\begin{equation} \label{ChargeContinuity}
%  \scalebox{1.0}{\includegraphics*{\picdir/ChargeContinuity}}  
    \nabla \cdot \vec{J} + \frac{\partial}{\partial t} \rho_{charge} = 0 
\end{equation}
Here, the medium is assumed linear and isotropic with homogeneous
permittivity, $\epsilon$, and permeability, $\mu$.
The charge density is denoted $\rho_{charge}$.
The electric and magnetic fields are denoted
$\vec{E}$ and $\vec{B}$, respectively. 
The current density, $J$, is assumed related to the electric field
through Ohm's law, $\vec{J}=\sigma \vec{E}$, where $\sigma$ denotes
the electrical conductivity.
In the quasistatic case, the electric field is approximated as
irrotational and Maxwell's system decouples
\[ 
%  \scalebox{1.0}{\includegraphics*{\picdir/irrotational}}  
 \nabla \times \vec{E}   = 0 
 \qquad \Rightarrow \qquad
 \nabla \cdot \sigma \vec{E}   = 0 
\]
For an arbitrary control volume, the Poynting vector defines the amount
of electromagnetic power, $Q_{em}$, which crosses the surface
\[
%  \scalebox{1.0}{\includegraphics*{\picdir/poyntingVector}}  
 Q_{em} = \int_{\text{surface}} 
                \left( \vec{E} \times \vec{B} \right) dA
\]
From the conservation of energy for an arbitrary control volume, the
mechanical energy and electromagnetic energy are coupled~\cite{pao1967fd}
\[
%  \scalebox{1.0}{\includegraphics*{\picdir/conserveNRG}}  
 \rho  c_p \frac{\partial u}{\partial t} + 
           \frac{\partial u_{em}}{\partial t} + 
  -\nabla  \cdot k \nabla u + \omega c_{blood} (u - u_a )
 = -Q_{em}(x,t) \\
\]
Here, the temperature is denoted $u$,
$\rho$  is the density of the continuum, which is liver tissue in
our case, and  $c_{blood}$ is the specific heat of blood. 
The perfusion coefficient, $\omega$, and thermal conductivity, $k$,
are assumed linear and
spatially homogeneous model coefficients. 
Combining the conservation of energy with Poynting's 
theorem~\cite{jackson1999cer} yields
\[
%  \scalebox{1.0}{\includegraphics*{\picdir/pennesQuasi}}  
 \rho  c_p \frac{\partial u}{\partial t} + 
           \frac{\partial u_{em}}{\partial t} + 
  -\nabla  \cdot k \nabla u + \omega c_{blood} (u - u_a )
 = \vec{E} \cdot \vec{J} + \frac{\partial u_{em}}{\partial t} 
\]
%From Poynting's theorem, the energy density of the
%electromagnetic field does not change
%\[
%\begin{split}
% \frac{\partial u_{EM}}{\partial t}  
%    & = 
%        - \nabla \cdot \frac{1}{\mu}\left( \vec{E} \times \vec{B} \right) 
%        - \vec{J} \cdot \vec{E}
% \\
%    &  = 
%        - \frac{1}{\mu}\left( \vec{B} \cdot \nabla \times \vec{E} 
%                            - \vec{E} \cdot \nabla \times \vec{B} \right) 
%        - \vec{J} \cdot \vec{E}
% \\
%    &  = 
%          \frac{1}{\mu} \vec{E} \cdot \left( \mu J \right) 
%        - \vec{J} \cdot \vec{E}
% \\
%    &  =  0 
%\end{split}
%\]
hence, the energy input into the tissue is
dissipated through the resistive heating.
\[
%  \scalebox{1.0}{\includegraphics*{\picdir/resistiveHeat}}  
Q_{RF}(x,t) = \vec{E} \cdot \vec{J} 
            = \sigma \vec{E} \cdot \vec{E} 
\]
The relationship of the voltage, $\phi$, as the scalar potential for the
electric field, $\nabla \phi = \vec{E}$, leads to the time dependent system
of coupled equations governing the temperature and voltage.
\begin{equation} \label{bioheatmodel}
%\scalebox{1.0}{\includegraphics*{\picdir/bioheatmodel}}  
\left.
\begin{split}
 \rho  c_p \frac{\partial u}{\partial t}
 & -\nabla  \cdot k \nabla u + \omega c_{blood} (u - u_a )
 = Q_{RF}(x,t) \\
 & Q_{RF}(x,t) = \sigma |\nabla \phi|^2, \qquad 
        \nabla \cdot \sigma \nabla \phi = 0
\end{split}
\right\}
\end{equation}
Dimensionally, the units of $Q_{RF}$ are consistent with power density.
\[
%  \scalebox{1.0}{\includegraphics*{\picdir/powerUnits}}  
\frac{S}{m} \frac{V^2}{m^2} 
  = \frac{\frac{A}{V}}{m} \frac{V^2}{m^2}  
  = \frac{A V}{m^3} 
  = \frac{ \frac{C}{s} \frac{J}{C} }{m^3} 
  = \frac{W}{m^3} 
\]
Neumann and Dirichlet boundary conditions are used on the surface of
the biological domain of interest, $\partial \Omega$, for the 
the temperature and voltage, respectively. 
\[
%\scalebox{1.0}{\includegraphics*{\picdir/tempvoltbc}}  
\left.
\begin{split}
 - k(u)      \nabla   u  \cdot \textbf{n}  & = 0 
\\
  \phi = 0
\end{split} 
\right\}
\qquad \text{on } \partial \Omega
\]
A summary of the 
constitutive data used is presented in Table~\ref{modeldata}.

\begin{table}[h]
\caption{Constitutive Data~\cite{Handbook05,stauffer2003paa,mcnichols2004mtb}}\label{modeldata}
\centering
\begin{tabular}{|c|c|c|c|c|c|c|c|c|} \hline
$\sigma$  &  $k$   &  $\rho$                &  $u_a$       &
$c_{blood}$                  &  $c_p$  & $A$ & $E_a$  & $R$  \\ \hline
 0.69 $\frac{S}{m}$  &  0.5  $\frac{W}{m K}$ &  1045 $\frac{kg}{m^3}$ &  308  $K$    &  3840 $ \frac{J}{kg \cdot K}$ &  3600 $ \frac{J}{kg \cdot K}$ 
  &  3.1e98 $ \frac{1}{s}$ &  6.28e5 $ \frac{J}{mol}$ &8.314 $\frac{J}{kg \; mol}$ \\ \hline
\end{tabular}
\end{table}


\paragraph{Acknowledgments}
%If you'd like to thank anyone, place your comments here
%and remove the percent signs.
The research in this paper was supported in part through the NIH
5T32CA119930-03 funding mechanism.  Canine data was obtained from
BioTex, Inc., under grants R43-CA79282, R44-CA79282, R43-AG19276.
The authors would also like to thank the
ITK~\cite{ITKSoftwareGuideSecondEdition}, Paraview~\cite{Paraview},
PETSc~\cite{petsc-manual}, libMesh~\cite{libMesh}, and
CUBIT~\cite{cubit} communities for providing truly enabling
software for scientific computation and visualization.  The initial
3D canine model in Figure~\ref{canineSetup} was provided by AddZero
and obtained from the official
Blender~\cite{roosendaal2000official} model repository.  Parameter
studies were performed using allocations  at the Texas Advanced
Computing Center. 

\end{document}
% end of file template.tex



