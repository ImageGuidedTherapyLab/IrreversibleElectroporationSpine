\documentclass{article} 
%=================================================
%  possible font errors: fi fl
%=================================================
\usepackage{amssymb,amsfonts,amsmath}
\usepackage{color,graphicx}
\usepackage[left=0.7in,right=0.7in,top=0.7in,bottom=0.7in]{geometry}
\newcommand{\eqn}[1]{(\ref{#1})}
%http://latex2rtf.sourceforge.net/latex2rtf_1_9_19.html#Conditional-Parsing
%Starting with LaTeX2RTF 1.9.18, there is a handy method for
%controlling which content should be processed by LaTeX or by
%LaTeX2RTF . Control is achieved using the standard \if facility of
%TeX. If you include the following line in the preamble of your document 
%
%     \newif\iflatextortf
%Then you will create a new \iflatextortf command in LaTeX . TeX
%sets the value of this to false by default. Now, LaTeX2RTF
%internally sets \iflatextortf to be true, and to ensure that this
%is always the case, LaTeX2RTF ignores the command
%\latextortffalse. This means that you can control how different
%applications process your document by
%
%     \iflatextortf
%     This code is processed only by latex2rtf
%     \else
%     This code is processed only by latex
%     \fi
%Note that \iflatextortf will only work within a section; you
%cannot use this command to conditionally parse code that crosses
%section boundaries. Also, it will only work on complete table or
%figure environments. Due to the mechanism used by LaTeX2RTF in
%processing these environments, at this time the only way to
%conditionally parse tables and figures is to include two complete
%versions of the environment in question, nested within an
%appropriate \iflatextortf structure.
%
\newif\iflatextortf

\iflatextortf 
%do nothing
\else  %pdflatex
\usepackage{boxedminipage,float}
\usepackage{wrapfig,setspace}
\newcommand{\picdir}{pdffig}
\fi

\usepackage[pdftex, plainpages=false, colorlinks=true, citecolor=black, filecolor=black, linkcolor=black, urlcolor=black]{hyperref}

%=================================================
\begin{document}

\title{\bf \Large
IRE Methods
}

\author{ D.~Fuentes$^1$
}

%\date{ \small
%$^1$The University of Texas M.D. Anderson Cancer Center,\\
%Department of Imaging Physics, Houston TX 77030, USA \\
%$^2$BioTex, Inc., Houston TX 77054, USA\\
%Email: \texttt{fuentesdt@gmail.com, jstafford@mdanderson.org}   \\
%% Webpage: \texttt{http://wiki.ices.utexas.edu/dddas}
%}

%\date{Received: April 2010 / Accepted: XXX }
% The correct dates will be entered by the editor


\maketitle

\paragraph{Keywords} IRE $\cdot$ 
                     Finite Element Modeling

{\color{red} what is the frequency in Hz of the applied IRE ? is the
frequency the same a RF ? how would electrical properties differ at the two
frequencies ? What frequency would B1 maps measure at ? }


Each IRE vertebral ablation 
delivered by the Nanoknife generator system (Angiodynamics, Queensbury, NY)
was mathematically modeled as
a quasi-static electric field induced by the voltage applied at the electrodes.
Similar to radiofrequency ablation~\cite{fuentesjvir},
the quasistatic equation for the voltage is
obtained from the Maxwell equations with an irrotational electric field. 
Resistive heating is not considered. 
The electrical conductivity is assumed piecewise constant across
distinct tissue within the spinal anatomy (21);
cerebrospinal fluid (CSF) 2.0 S/m; spinal cord 0.23
S/m; skeletal muscle 0.1 S/m; bone 0.02 S/m; fat 0.012 S/m.
A finite mixture model~\cite{avants2011} was applied to CT images to
classify the tissue types. Nerve roots were manually located.


A finite element method was used to numerically simulate the
electric field within the tissue. The finite element mesh was discretized to
conform to the geometry of the applicator.
A single template FEM mesh was used to investigate each dataset.
The applicator was located on the CT image of each dataset to register the computational
domain to the imaging.
To ensure convergence of the numerical method, the spatial mesh
discretization was refined until there was less than a 0.5\% difference
in solution between refinements.
Boundary conditions at the 
charged and grounded electrode surface were taken to
be the applied voltage (2700, 1000, 833, 667, and 500 V) and zero, respectively.
The domain used to solve for the models was large enough that electric
insulation boundary conditions are appropriate. 
The average
and maximum electric field exposure of the spinal cord and exiting nerve
root were calculated.  


%%%%%%%%%%%%%%%%%%%%%%%%%%%%%%%%%%%%%%%%%%%%%%
\pagebreak
\section{Electric conductivity $\sigma$ estimates}
%%%%%%%%%%%%%%%%%%%%%%%%%%%%%%%%%%%%%%%%%%%%%%

Faraday's law in integral form
and Ampere's law in differential form may be combined to yeild an
expression for the complex permittivity, $\kappa$~\cite{katscher2009determination}.
\begin{equation}\label{FaradayAmpere}
  \frac{\oint_{\partial A} \nabla \times \vec{H}  \cdot dl }{
        \mu \omega^2  \int_A \vec{H} \cdot d\vec{a}
       }
   \approx
     \kappa  = \epsilon  - i \; \sigma / \omega
\qquad
  \vec{H} = H_x \hat{x}   
          + H_y \hat{y}   
          + H_z \hat{z}   
\end{equation}

The magnetic field applied by the transmit coil is denoted $\vec{H}$,	
$\omega$ is the Larmor frequency, and $\mu$ is the permeability.
\begin{figure}[h]
\centering\scalebox{0.2}{\includegraphics*{\picdir/ImagingPlane.png}}
\caption{
Coronal Imaging Plane~\cite{katscher2009determination}.
}\label{Fig:ImagingPlane}
\end{figure}
A non-axial plane such as the 
coronal plane, $y=$ constant, is perferred because the axial component,
$H_z$ is not measurable. 
For reference coronal imaging plane with respect to a 
right hand coordinate system is shown in Figure~\ref{Fig:ImagingPlane},
In this case \eqn{FaradayAmpere} may be reduced
 to an expression for the conductivity, $\sigma$.
\begin{equation}
  \frac{\oint_{\partial A_{xz}}
        \left\{
          \partial_y \underbrace{H_z}_{= 0 \text{ by assumption}} - 
          \partial_z             H_y
              , 
           \partial_x H_y - \partial_y H_x
        \right\}  \cdot dl }{
        \mu \omega^2  \int_{A_{xz} } H_y \cdot \; dx dz
       }
   \approx
     \kappa  = \epsilon  - i \; \sigma / \omega
\qquad
\Rightarrow
\qquad
\sigma \approx
  \frac{\oint_{\partial A_{xz}}
           \partial_x H_y - \partial_y H_x
                  \; ds  }{
        \mu \omega  \int_{A_{xz} } H_y \cdot \; dx dz
       }
\end{equation}
The integrals yeild a scalar valued average estimate of conduction, $\sigma \in
\mathbb{R}$ with the ROI of interest, $A$.

Paraphasing~\cite{katscher2009determination}: 
"The use of a birdcage-type RF coil is advantageous,
i.e., a quadrature body coil (QBC) or corresponding
head coil. A birdcage coil allows choosing the coil polarization
during transmission and reception. In particular,
the structure of the birdcage
coil imposes that transmit sensitivity,
$H_+ = \frac{H_x + i H_y}{2}$, is substantially larger than receive sensitivity,
$H_- = \frac{H_x - i H_y}{2}$.
The determination of $H_+$, is the so-called B1 mapping. 
A double angle method~\cite{insko1993mapping} is appropriate.
However, these approaches only describe the determination of the modulus 
$|H_+|$, from the signal ratio observed at the two flip angles $\theta_i$ .
\begin{equation}
 |H_+| = \frac{\cos^{-1}\left(\frac{1}{2\;r}\right)}{\omega t}
  \qquad
  r = \frac{1}{2 \; \cos \theta_1} = \frac{\sin \theta_1}{\sin \theta_2}
  \qquad
  \theta_2 = 2 \; \theta_1
\end{equation}
The use of a switched birdcage coil allows also the estimation of the B1
phase."




\begin{thebibliography}{10}
\bibitem[11]{insko1993mapping}
EK~Insko and L~Bolinger.
\newblock Mapping of the radiofrequency field.
\newblock {\em Journal of Magnetic Resonance, Series A}, 103(1):82--85, 1993.

\bibitem[12]{katscher2009determination}
Ulrich Katscher, Tobias Voigt, Christian Findeklee, Peter Vernickel, Kay
  Nehrke, and O~Dossel.
\newblock Determination of electric conductivity and local sar via b1 mapping.
\newblock {\em Medical Imaging, IEEE Transactions on}, 28(9):1365--1374, 2009.

\end{thebibliography}

%%%%%%%%%%%%%%%%%%%%%%%%%%%%%%%%%%%%%%%%%%%%%%
\pagebreak
\section{notes}
%%%%%%%%%%%%%%%%%%%%%%%%%%%%%%%%%%%%%%%%%%%%%%


Numerical results also supported this decision. A field strength of xxx
V/cm was observed in the nerve roots when 833 V/cm was applied at the
electrodes. At 667 V/cm, the average field strength in the nerve roots
predicted was xxx V/cm and below the known threshold for nerve damage. 


Purpose

To determine the energy threshold for spinal cord damage during irreversible
electroporation (IRE) in the porcine spine.
Materials and Methods

Four terminal animals underwent computed tomography-guided IRE at five
vertebral levels each, using different applied voltages (2700, 1000, 833,
667, and 500 V). The electrode was placed through the lateral recess of the
vertebral body. MRI imaging was performed 6-hrs after IRE. Histopathology of
the spinal cord, nerve root, and ganglion was assessed using light
microscopy. From the results of these animals, the electric field strength
that demonstrated no imaging or histopathologic evidence of injury to the
neural structures was selected for use in a survival study. Four pigs
underwent IRE of 4 vertebral levels each using 667 V, were imaged with MRI,
and survived for 7 days. Imaging and histopathologic data were analyzed. A
finite element method was used to numerically simulate the electric field
within the tissue. The finite element mesh was discretized to conform to the
geometry of the IRE probe. Heterogeneous electrical conductivity tissue
properties were assumed for the cerebrospinal, spinal cord, skeletal muscle,
bone, and fat.
Results

The mean distance between the IRE probe and the spinal cord was 1.3 +/- 0.9
mm. MRI and histopathology showed no evidence of spinal cord injury in the
terminal animals irrespective of electric field strength applied.
Histopathologic evaluation showed injury to one nerve root or ganglion when
voltage applied was > 833 V. Thus, 667 V, which was also supported by the
mathematical model, was used for the survival study. Average field strengths
of 130 V/cm and 96 V/cm were observed in the nerve roots when 833 V and 667
V were applied, respectively. In the survival animals, MRI showed no
evidence of injury to the spinal cord and none were clinically symptomatic.
Histopathology is pending.
Conclusion

While the spinal cord seems immune to the effects of IRE, injury to the
nerve roots or ganglia may be the limiting factor for use in the vertebral
body. Incorporating the use of mathematical models to set inverse treatment
parameters may be helpful for treatment planning.



semi-automatic methods were used for segmentation. 
a mixture model was applied followed by  manual cleanup
manually segment and
extract key anatomic features (spinal
cord, vertebra, paraspinal muscle



 
same electrode hooked up to angiodynamics system as well as rfa system for
impedance measurement ? 
electrode (Covidien, ACT2020, Single Electrode Kit, 20 cm - 2 cm )



%FIXME: was it (call tech support)
%     - 2 electrodes at the applied voltage 
%             - or -
%     - 1 electrode at the voltage and one grounded with insulating BC


%===================================================================
\section{Methods}
%===================================================================
{\color{red}

Several of the thermal techniques
use electrical pulses and the Joule heating effect to elevate the
temperature of
the targeted tissue. For example, radiofrequency (RF) ablation requires an
active electrode to be introduced into the undesirable tissue and an
alternating
current with high frequency of up to 500 kHz and high voltage, up to several
(four) kV, to be used to heat the tissue to coagulation (3). ~\cite{Davalos2005}
Traditional Joule heating
employs electrodes inserted in tissue and DC or AC currents, e.g. (4).
Microwave ablation uses electromagnetic fields in the MHz frequency domain
(5) for thermal ablation. A recent review of these techniques applied to
liver
cancer gives a thorough analysis of each (6).




The most commonly
used voltage parameters for ECT are eight 100-μs
pulses at a 1-Hz frequency, and 1300 V is often applied
across the two electrodes


In this study~\cite{Miklavcic2000} correlating electroporation experiments with
mathematical modeling, they found the electric field threshold
for reversible electroporation is 362 ± 21 V cm−1 and
the threshold for irreversible electroporation is 637 ± 43 V
cm−1 for rat liver tissue using eight 100-μs pulses at a frequency
of 1 Hz. Therefore, in our analysis, we have taken
an electric field of 360 V cm−1 to represent the delineation
between no electroporation and reversible electroporation,
and 680 V cm−1 to represent the delineation between reversible
and irreversible electroporation.

}




%The 32 slices that made up the 3D image were loaded into ITK-SNAP (Penn
%Image Computing and Science Laboratory (PICSL), Dept. of Radiology,
%University of Pennsylvania) an application for segmenting 3D images.
%Once generated from the 3D image the segmented brain data was saved and
%loaded into Cubit (Sandia National Laboratories Albuquerque, New Mexico)
%to be meshed.  Cubit generated a mesh of hexahedral elements throughout
%the 3D volume of the brain. 

%Following the creation of the brain mesh a second mesh object
%representing the laser source was constructed and placed at the
%appropriate location within the simulated brain. 
The thermal effects are expected to be confined to a region near
the applicator. Consequently, the boundary conditions of a
conformal canine specific 3D volumetric hexahedral FEM mesh, shown
in Figure~\ref{canineSetup} are expected to have no effect on the
results.  



\textcolor{blue}{ % comments for review 1
The ROI for each canine, $\Omega \subset \mathbb{R}^3$ (FOV=
1.95x2.38cm $\pm$ .21x.19cm) 
was chosen large enough to encompass the
extent of the heating region but small enough to minimize bias of
the error metric \eqn{weightedL2Norm} where no  heating occurred.
Individual regions of the brain were not segmented due to the
localized nature of the heating study.
To ensure convergence, five FEM mesh resolutions were considered for
each ROI (min/median/max \# elements = 896/3693/11978, 
          min/median/max \# nodes    = 1349/4674/13874).
Within the ROI, the mesh size element diameter ranged between
[0.5mm,2mm]. As a reference, 1mm is on the order of the pixel size.
}


%%%%%%%%%%%%%%%%%%%%%%%%%%%%%%%%%%%%%%%%%%%%%%
\subsubsection{Constitutive Data}
%%%%%%%%%%%%%%%%%%%%%%%%%%%%%%%%%%%%%%%%%%%%%%

The bio-thermal and optical parameters were obtained from
literature~\cite{Handbook05,Welch95,duck1990} and were modeled as
homogeneous throughout the delivery region of interest,
Table~\ref{modeldata}.  An average value of the range of
scattering, $\mu_s$, and absorption, $\mu_a$, values quoted in the
literature was used.  A range of perfusion values
$\omega$=0.0,3.0,6.0,12.0$[\frac{kg}{s m^3}]$ were simulated to
capture physically realistic values of brain tissue perfusion found
in literature and as a first order study of temperature dependent
perfusion.  Empirical models of the temperature dependence of the
constitutive data~\cite{pegau1997absorption,Duggan00,Handbook05},
$k(u)$, $\omega(u)$, $\mu_a(u)$, will be considered in future
validation studies.  Statistics were collected comparing the
thermal imaging data for each animal against each permutation of
laser model, cooling model, and perfusion value, a total of 64
simulations at each mesh resolution considered. 

\begin{table}[h]
\caption{Constitutive Data~\cite{Handbook05,Welch95,duck1990}}\label{modeldata}
\centering 
\begin{tabular}{|c|c|c|c|c|c|c|c|} \hline 
$k $ $ \frac{W}{ m \cdot K}$ & $\omega$ $\frac{kg}{m^3 s}$ &  $g$  &  $\mu_s$ $\frac{1}{cm}$  &  $\mu_a$  $\frac{1}{cm}$   &  $\rho$ $\frac{kg}{m^3}$ &   $c_{blood}$ $ \frac{J}{kg \cdot K}$ &  $c_p$ $\frac{J}{kg \cdot K}$ \\ \hline
          0.527              &     0.0,3.0,6.0,12.0        & 0.862 &        47.0              &       0.45                 &  1045                        &            3840                      &                  3600          \\ \hline
\end{tabular}
\end{table}


%%%%%%%%%%%%%%%%%%%%%%%%%%%%%%%%%%%%%%%%%%%%%%%%%%%%%%%%%%%%%%%%%%%
\subsubsection{Theoretical Probability Density}
%%%%%%%%%%%%%%%%%%%%%%%%%%%%%%%%%%%%%%%%%%%%%%%%%%%%%%%%%%%%%%%%%%%
The theoretical probability density, $\Theta$, represents the
mathematical model predicted relationship between the parameter
space, $\mathbb{M}$, and the data space, $\mathbb{D}$. 
For a well defined mathematical model in which a unique solution
exists, the theoretical probability density may be taken as a delta
functional.
This assumption explicitly ignores any potential modeling error
and enforces the notion that the event of a temperature field not
predicted by the theory has zero probability of occurring.
The nonlinear mapping from parameter space to data space, 
$u(\vec{m}) \in \mathbb{D}$, 
is obtained as a solution to the Pennes bioheat
transfer model~\cite{Pennes1948} coupled with the
diffusion theory of light interaction with tissue~\cite{Welch95}. 
\[ 
 \rho  c_p \frac{\partial u}{\partial t}
 -\nabla \cdot ( {k(\textbf{x})} \nabla u) 
 + {\omega(\textbf{x})} c_{blood} (u - u_a )
 = {\mu_a(\textbf{x})} \varphi(m,\textbf{x},t)  \quad \text{in } \Omega
 \]
\[
   - {k(\textbf{x})} \nabla u \cdot \textbf{n} = 0
        \qquad \text{on } \partial \Omega
\qquad
   u(\textbf{x},0) = u^0 \qquad \text{in } \Omega
\]
The initial temperature field, $u(\textbf{x},0)=u^0$, 
is taken as the measured baseline body
temperature.  The density of the continuum is denoted $\rho$, the
specific heat of blood is denoted $c_{b} \left[\frac{J}{kg \cdot
K}\right]$, $k(\textbf{x})$ denotes the thermal conductivity, and
$\omega(\textbf{x})$ denotes perfusion.
Zero heat flux Neumann data is assumed on the boundary, $\partial \Omega$.
Although likely to result in optical parameter estimation 
error due to the strong scattering assumptions~\cite{carp2004radiative},
the classical
spherically symmetric isotropic standard diffusion approximation (SDA)
to the transport equation of
light within a laser-irradiated tissue~\cite{Welch95} is used as the kernel
in modeling the laser source term.  
The SDA was used in this
feasibility study for its computational simplicity in managing
the algorithmic complexity of computing the gradient and Hessian of the misfit function, 
Section~\ref{GradientHessianEvaluation}.
The optical-thermal response to the laser source,
$q_{laser} = {\mu_a(\textbf{x})} \varphi(m,\textbf{x},t)$,
is modeled as 
\[
\begin{split}
   \varphi(m,\textbf{x},t) & = 
    \int_{U_{tip}}
 \frac{P(t)  }{\text{Vol}(U_{tip}) }  
   \frac{3\mu_{tr} \exp(-\mu_{eff} \| \textbf{x} -\mathbf{ \xi}\|) }
      {4\pi \| \textbf{x}-\mathbf{ \xi}\|} \; d\mathbf{\xi}
\quad
\\
 & \hspace{2in}   x \in U \backslash U_{tip}
\\
 & 
  \mu_{tr}   = {\mu_a(\textbf{x}) } +  
               {\mu_s(\textbf{x})} (1-g)
\qquad
  \mu_{eff}  = \sqrt{3 {\mu_a(\textbf{x})} \mu_{tr}}
\end{split}
\]
$P(t) $ is the
laser power as a function of time, $\mu_a$ and $\mu_s $ are laser
coefficients related to laser wavelength and give probability of
absorption and scattering of photons, respectively. The anisotropic
factor is denoted $g$  
and $\|x-\xi\|$ denotes the distance from the laser photon source, $U_{tip}$.
Active cooling of the applicator is modeled by holding the temperature
in this region fixed through Dirichlet boundary conditions at the ambient temperature
\cite{fuentesetal11a}.

Given an initial temperature field, $u^0$, a set of model parameters,
$
m \equiv ( k(\textbf{x}), \omega(\textbf{x}), 
           \mu_a(\textbf{x}),\mu_s(\textbf{x})) \in \mathcal{M}
$, and the appropriate boundary condition, a model prediction of
the temperature field, $u(x,t)$, may be obtained.
Discretizing the temperature with a Galerkin expansion at each 
time step, the data space is represented by
the vector space of Galerkin
coefficients for the thermal field.
Similarly, a Galerkin discretization is  used for the model
parameters. The model space, $\mathbb{M}$, is the space of model
parameters for which a unique solution is well defined.
\[
\begin{split}
u(x,t^k) = \sum_j^{N_{dof}}  d_j^k  \phi_j(x) 
\qquad 
k = 1, 2, ..  N_{step}
\qquad
\qquad
\mathbb{D} = \mathbb{R}^{N_{dof}*N_{step}}
\\
m(x) = \sum_j^{N_{model}}  m_j  \psi_j(x) 
\qquad
\qquad
\mathbb{M} \subset \mathbb{R}^{N_{model}}
\end{split}
\]
Here $N_{dof}$ and $N_{model}$ are the spatial temperature degrees of freedom and the parameter
degrees of freedom, respectively. The number of time steps in the discretization is denoted $N_{step}$.
In this application, an element of the data space, $\vec{d} \in
\mathbb{D}$, is a set of Galerkin coefficients that may represent
either an experimentally observed MRTI quantity or a quantity
predicted by the bioheat transfer equation.  

%%%%%%%%%%%%%%%%%%%%%%%%%%%%%%%%%%%%%%%%%%%%%%
\subsection{Discretization and Constitutive Data}~\label{DiscretizationConstitutive}
%%%%%%%%%%%%%%%%%%%%%%%%%%%%%%%%%%%%%%%%%%%%%%

The finite element model of the \textit{in vivo} TVT inoculated canine
MRgLITT is shown in Figure~\ref{nanodog}. The insertion of the
applicator into the tissue was modeled using a mesh that consisted of
distinct regions for the applicator, healthy tissue, and cancer burden.
A quadrilateral mesh was extruded axially along the applicator to create
the base of the hexahedral finite element mesh. The mesh for the tissue
conforms to the boundary of the application and extends to the boundary
of the brain to ensure that the boundary does not influence the heating.
The entire mesh consists of $N_{dof}=$47,941 total nodes (grid points).
At each time step of the Crank-Nicolson scheme used, the temperature
across the applicator was held constant to model the effect of the
room-temperature cooling fluid which protects the laser fiber during
heating. The degrees of freedom across the applicator were held at
21$^o$C by treating the corresponding degrees of freedom as Dirichlet
boundary data. The bioheat transfer model temperature prediction is not
seen to be very sensitive to the optical scattering~\cite{Fengetal07}. 
Only optical absorption and thermal conductivity
parameter estimation is considered in this work.
The combined thermal conductivity and optical
absorption estimation consisted of $N_{model}=$4,570
spatially varying parameters. 
The remaining input parameters are assumed deterministic.
 The laser power profile used
during the therapy is shown in Figure~\ref{PowerProfile}. The test
pulse, 3W for 30 seconds, and main therapeutic pulse, 3.5W for 3min, are
shown. 
%There are many combinations active power durations and magnitude,
%including the data available, that may provide the significant heating
%and fluence needed to recover the optical parameters and subsequently
%predict the therapeutic heating accurately. 
The amount, duration, and magnitude of heating data needed to accurately
recover the optical parameters during the ``test" pulse of the
experimental data and subsequent critical evaluation of accuracy of the
calibrated computer model prediction during the therapeutic delivery is
considered outside the scope of this work. The current work seeks to
demonstrate the potential of parameter estimation, hence, parameter
estimation for verifying uptake and potentially planning the procedure
is considered for the time interval of the strongest fluence signal for
the data acquired. Data acquired during the time window of the
therapeutic heating and subsequent cooling, $\delta t = [204,504]$,
$N_{step} = 300$ seen
in Figure~\ref{PowerProfile}, is used to demonstrate the feasibility of
recovering the 3D spatially varying optical distribution.  The
Crank-Nicolson scheme was used to propagate the temperature field at one
second time steps.  Table~\ref{modeldata} summarizes the constitutive
data used.

\begin{table}[h]
\caption{Constitutive Data~\cite{Handbook05,Welch95,duck1990}}\label{modeldata}
\centering 
\begin{tabular}{|c|c|c|c|c|c|c|c|} \hline 
$[k^{lb},k^{ub}]$ $ \frac{W}{ m \cdot K}$ & $\omega$ $\frac{kg}{m^3 s}$ &  $g$  &  $\mu_s$ $\frac{1}{m}$  &  $[\mu_a^{lb},\mu_a^{ub}]$  $\frac{1}{m}$   &  $\rho$ $\frac{kg}{m^3}$ &   $c_{blood}$ $ \frac{J}{kg \cdot K}$ &  $c_p$ $\frac{J}{kg \cdot K}$ \\ \hline
          [0.1, 0.7]                    &             9.0             & 0.88  &       34.0e3              &     [60.0, 600.0]                             &  1045                        &            3840                      &                  3600          \\ \hline
\end{tabular}
\end{table}

The field was optimized in a 1.5cm diameter spherical ROI about
the laser applicator centroid.



%========================================================================
\paragraph{\textbf{{\large Appendix}}}
%========================================================================

The equations governing the resistive heating that provides the thermal
source for the bioheat equation~\eqn{bioheatmodel} may be obtained from
Maxwell's equation~\cite{demkowicz2006cha} using Faraday's
law~\eqn{FaradayLaw}, Ampere's law~\eqn{AmpereLaw}, and the continuity
of free charge~\eqn{ChargeContinuity}.
\begin{equation} \label{FaradayLaw}
%  \scalebox{1.0}{\includegraphics*{\picdir/FaradayLaw}}  
 \nabla \times \vec{E} = -\frac{\partial}{\partial t} \vec{B}
\end{equation}
\begin{equation} \label{AmpereLaw}
%  \scalebox{1.0}{\includegraphics*{\picdir/AmpereLaw}}  
 \nabla \times \vec{B} = 
        \mu \vec{J} + \mu \epsilon \frac{\partial}{\partial t} \vec{E}
\end{equation}
\begin{equation} \label{ChargeContinuity}
%  \scalebox{1.0}{\includegraphics*{\picdir/ChargeContinuity}}  
    \nabla \cdot \vec{J} + \frac{\partial}{\partial t} \rho_{charge} = 0 
\end{equation}
Here, the medium is assumed linear and isotropic with homogeneous
permittivity, $\epsilon$, and permeability, $\mu$.
The charge density is denoted $\rho_{charge}$.
The electric and magnetic fields are denoted
$\vec{E}$ and $\vec{B}$, respectively. 
The current density, $J$, is assumed related to the electric field
through Ohm's law, $\vec{J}=\sigma \vec{E}$, where $\sigma$ denotes
the electrical conductivity.
In the quasistatic case, the electric field is approximated as
irrotational and Maxwell's system decouples
\[ 
%  \scalebox{1.0}{\includegraphics*{\picdir/irrotational}}  
 \nabla \times \vec{E}   = 0 
 \qquad \Rightarrow \qquad
 \nabla \cdot \sigma \vec{E}   = 0 
\]
For an arbitrary control volume, the Poynting vector defines the amount
of electromagnetic power, $Q_{em}$, which crosses the surface
\[
%  \scalebox{1.0}{\includegraphics*{\picdir/poyntingVector}}  
 Q_{em} = \int_{\text{surface}} 
                \left( \vec{E} \times \vec{B} \right) dA
\]
From the conservation of energy for an arbitrary control volume, the
mechanical energy and electromagnetic energy are coupled~\cite{pao1967fd}
\[
%  \scalebox{1.0}{\includegraphics*{\picdir/conserveNRG}}  
 \rho  c_p \frac{\partial u}{\partial t} + 
           \frac{\partial u_{em}}{\partial t} + 
  -\nabla  \cdot k \nabla u + \omega c_{blood} (u - u_a )
 = -Q_{em}(x,t) \\
\]
Here, the temperature is denoted $u$,
$\rho$  is the density of the continuum, which is liver tissue in
our case, and  $c_{blood}$ is the specific heat of blood. 
The perfusion coefficient, $\omega$, and thermal conductivity, $k$,
are assumed linear and
spatially homogeneous model coefficients. 
Combining the conservation of energy with Poynting's 
theorem~\cite{jackson1999cer} yields
\[
%  \scalebox{1.0}{\includegraphics*{\picdir/pennesQuasi}}  
 \rho  c_p \frac{\partial u}{\partial t} + 
           \frac{\partial u_{em}}{\partial t} + 
  -\nabla  \cdot k \nabla u + \omega c_{blood} (u - u_a )
 = \vec{E} \cdot \vec{J} + \frac{\partial u_{em}}{\partial t} 
\]
%From Poynting's theorem, the energy density of the
%electromagnetic field does not change
%\[
%\begin{split}
% \frac{\partial u_{EM}}{\partial t}  
%    & = 
%        - \nabla \cdot \frac{1}{\mu}\left( \vec{E} \times \vec{B} \right) 
%        - \vec{J} \cdot \vec{E}
% \\
%    &  = 
%        - \frac{1}{\mu}\left( \vec{B} \cdot \nabla \times \vec{E} 
%                            - \vec{E} \cdot \nabla \times \vec{B} \right) 
%        - \vec{J} \cdot \vec{E}
% \\
%    &  = 
%          \frac{1}{\mu} \vec{E} \cdot \left( \mu J \right) 
%        - \vec{J} \cdot \vec{E}
% \\
%    &  =  0 
%\end{split}
%\]
hence, the energy input into the tissue is
dissipated through the resistive heating.
\[
%  \scalebox{1.0}{\includegraphics*{\picdir/resistiveHeat}}  
Q_{RF}(x,t) = \vec{E} \cdot \vec{J} 
            = \sigma \vec{E} \cdot \vec{E} 
\]
The relationship of the voltage, $\phi$, as the scalar potential for the
electric field, $\nabla \phi = \vec{E}$, leads to the time dependent system
of coupled equations governing the temperature and voltage.
\begin{equation} \label{bioheatmodel}
%\scalebox{1.0}{\includegraphics*{\picdir/bioheatmodel}}  
\left.
\begin{split}
 \rho  c_p \frac{\partial u}{\partial t}
 & -\nabla  \cdot k \nabla u + \omega c_{blood} (u - u_a )
 = Q_{RF}(x,t) \\
 & Q_{RF}(x,t) = \sigma |\nabla \phi|^2, \qquad 
        \nabla \cdot \sigma \nabla \phi = 0
\end{split}
\right\}
\end{equation}
Dimensionally, the units of $Q_{RF}$ are consistent with power density.
\[
%  \scalebox{1.0}{\includegraphics*{\picdir/powerUnits}}  
\frac{S}{m} \frac{V^2}{m^2} 
  = \frac{\frac{A}{V}}{m} \frac{V^2}{m^2}  
  = \frac{A V}{m^3} 
  = \frac{ \frac{C}{s} \frac{J}{C} }{m^3} 
  = \frac{W}{m^3} 
\]
Neumann and Dirichlet boundary conditions are used on the surface of
the biological domain of interest, $\partial \Omega$, for the 
the temperature and voltage, respectively. 
\[
%\scalebox{1.0}{\includegraphics*{\picdir/tempvoltbc}}  
\left.
\begin{split}
 - k(u)      \nabla   u  \cdot \textbf{n}  & = 0 
\\
  \phi = 0
\end{split} 
\right\}
\qquad \text{on } \partial \Omega
\]
A summary of the 
constitutive data used is presented in Table~\ref{modeldata}.

\begin{table}[h]
\caption{Constitutive Data~\cite{Handbook05,stauffer2003paa,mcnichols2004mtb}}\label{modeldata}
\centering
\begin{tabular}{|c|c|c|c|c|c|c|c|c|} \hline
$\sigma$  &  $k$   &  $\rho$                &  $u_a$       &
$c_{blood}$                  &  $c_p$  & $A$ & $E_a$  & $R$  \\ \hline
 0.69 $\frac{S}{m}$  &  0.5  $\frac{W}{m K}$ &  1045 $\frac{kg}{m^3}$ &  308  $K$    &  3840 $ \frac{J}{kg \cdot K}$ &  3600 $ \frac{J}{kg \cdot K}$ 
  &  3.1e98 $ \frac{1}{s}$ &  6.28e5 $ \frac{J}{mol}$ &8.314 $\frac{J}{kg \; mol}$ \\ \hline
\end{tabular}
\end{table}


\paragraph{Acknowledgments}
%If you'd like to thank anyone, place your comments here
%and remove the percent signs.
The research in this paper was supported in part through the NIH
5T32CA119930-03 funding mechanism.  Canine data was obtained from
BioTex, Inc., under grants R43-CA79282, R44-CA79282, R43-AG19276.
The authors would also like to thank the
ITK~\cite{ITKSoftwareGuideSecondEdition}, Paraview~\cite{Paraview},
PETSc~\cite{petsc-manual}, libMesh~\cite{libMesh}, and
CUBIT~\cite{cubit} communities for providing truly enabling
software for scientific computation and visualization.  The initial
3D canine model in Figure~\ref{canineSetup} was provided by AddZero
and obtained from the official
Blender~\cite{roosendaal2000official} model repository.  Parameter
studies were performed using allocations  at the Texas Advanced
Computing Center. 

%\bibliographystyle{plainnat}
\bibliographystyle{plain}
\bibliography{ireMethods}

\end{document}



