\documentclass{article} 
%=================================================
%  possible font errors: fi fl
%=================================================
\usepackage{amssymb,amsfonts,amsmath}
\usepackage{color,graphicx}
\usepackage[left=1.0in,right=1.0in,top=1.0in,bottom=1.0in]{geometry}
\newcommand{\eqn}[1]{(\ref{#1})}
%http://latex2rtf.sourceforge.net/latex2rtf_1_9_19.html#Conditional-Parsing
%Starting with LaTeX2RTF 1.9.18, there is a handy method for
%controlling which content should be processed by LaTeX or by
%LaTeX2RTF . Control is achieved using the standard \if facility of
%TeX. If you include the following line in the preamble of your document 
%
%     \newif\iflatextortf
%Then you will create a new \iflatextortf command in LaTeX . TeX
%sets the value of this to false by default. Now, LaTeX2RTF
%internally sets \iflatextortf to be true, and to ensure that this
%is always the case, LaTeX2RTF ignores the command
%\latextortffalse. This means that you can control how different
%applications process your document by
%
%     \iflatextortf
%     This code is processed only by latex2rtf
%     \else
%     This code is processed only by latex
%     \fi
%Note that \iflatextortf will only work within a section; you
%cannot use this command to conditionally parse code that crosses
%section boundaries. Also, it will only work on complete table or
%figure environments. Due to the mechanism used by LaTeX2RTF in
%processing these environments, at this time the only way to
%conditionally parse tables and figures is to include two complete
%versions of the environment in question, nested within an
%appropriate \iflatextortf structure.
%
\newif\iflatextortf

\iflatextortf 
%do nothing
\else  %pdflatex
\usepackage{boxedminipage,float}
\usepackage{wrapfig,setspace}
\newcommand{\picdir}{pdffig}
\fi

\usepackage[pdftex, plainpages=false, colorlinks=true, citecolor=black, filecolor=black, linkcolor=black, urlcolor=black]{hyperref}

%=================================================
\begin{document}

\title{\bf \Large
IRE Methods
}

\author{ D.~Fuentes$^1$
}

%\date{ \small
%$^1$The University of Texas M.D. Anderson Cancer Center,\\
%Department of Imaging Physics, Houston TX 77030, USA \\
%$^2$BioTex, Inc., Houston TX 77054, USA\\
%Email: \texttt{fuentesdt@gmail.com, jstafford@mdanderson.org}   \\
%% Webpage: \texttt{http://wiki.ices.utexas.edu/dddas}
%}

%\date{Received: April 2010 / Accepted: XXX }
% The correct dates will be entered by the editor


\maketitle

\paragraph{Keywords} Bioheat Transfer $\cdot$ 
                     %MR Thermal Imaging $\cdot$ 
                     Laser Tissue Interaction $\cdot$ 
                     Finite Element Modeling

%===================================================================
\section{Methods}
%===================================================================

%The 32 slices that made up the 3D image were loaded into ITK-SNAP (Penn
%Image Computing and Science Laboratory (PICSL), Dept. of Radiology,
%University of Pennsylvania) an application for segmenting 3D images.
%Once generated from the 3D image the segmented brain data was saved and
%loaded into Cubit (Sandia National Laboratories Albuquerque, New Mexico)
%to be meshed.  Cubit generated a mesh of hexahedral elements throughout
%the 3D volume of the brain. 

%Following the creation of the brain mesh a second mesh object
%representing the laser source was constructed and placed at the
%appropriate location within the simulated brain. 
The thermal effects are expected to be confined to a region near
the applicator. Consequently, the boundary conditions of a
conformal canine specific 3D volumetric hexahedral FEM mesh, shown
in Figure~\ref{canineSetup} are expected to have no effect on the
results.  Thus, a single template FEM mesh was used to investigate
each canine data set. Magnitude images were used to manually
register the mesh template to the thermal imaging data.  The
objective of this study was to investigate modeling error in an ROI
about the active tip of the applicator.  

% full mesh resolution
%          elements    nodes
%mesh[0]     20069     22148 
%mesh[1]     27041     29568
%mesh[2]     47785     51366
%mesh[3]     52232     56565

% summary of mesh resolution across ROI
%                   meshlowerres    meshlores    meshnormres     mesh hires       original 
%(element,nodes)      mesh[0]        mesh[1]        mesh[2]        mesh[3]        mesh[4]    
%     ROI[1]      ( 2002, 2580)  ( 2678, 3388)  ( 8127, 9394)  ( 4200, 5200)  (10603,12324)
%     ROI[2]      ( 1513, 2076)  ( 1975, 2677)  ( 6665, 7953)  ( 3341, 4227)  ( 8091, 9741)
%     ROI[3]      ( 2393, 3029)  ( 3131, 3951)  ( 9013,10352)  ( 4896, 5974)  (11978,13874)
%     ROI[4]      (  896, 1349)  ( 1225, 1806)  ( 4045, 5120)  ( 2629, 3481)  ( 5523, 6989)
%
%original ROI data for mesh[0]
% ROI 1: # elements 10603  # nodes 12324  FOV 2.0cm x 2.1cm x 1.5cm
% ROI 2: # elements 8091   # nodes 9741   FOV 1.6cm x 2.5cm x 1.3cm
% ROI 3: # elements 11978  # nodes 13874  FOV 2.1cm x 2.6cm x 1.6cm
% ROI 4: # elements 5523   # nodes 6989   FOV 2.1cm x 2.3cm x 0.8cm
%elems = [10603,8091,11978,5523]  
%nodes = [12324,9741,13874,6989]  
%elems = [ 2002,2678,8127,4200,10603, 1513,1975,6665,3341, 8091, 2393,3131,9013,4896,11978, 896,1225,4045,2629, 5523 ]
%nodes = [2580, 3388, 9394, 5200,12324, 2076, 2677, 7953, 4227, 9741 , 3029 , 3951,10352, 5974,13874 , 1349, 1806 , 5120, 3481, 6989]
%
%xroi  = [2.0  ,1.6 , 2.1 , 2.1]  
%yroi  = [2.1  ,2.5 , 2.6 , 2.3]  
%print mean(elems) , std (elems)
%print mean(nodes) , std (nodes)
%print mean(xroi ) , std (xroi )
%print mean(yroi ) , std (yroi )
%>>> print mean(elems) , std (elems)
%9048.75 2467.00206475
%>>> print mean(nodes) , std (nodes)
%10732.0 2617.19114701
%>>> print mean(xroi ) , std (xroi )
%1.95 0.206155281281
%>>> print mean(yroi ) , std (yroi )
%2.375 0.192028643697


%>>> from numpy import mean
%>>> from numpy import std
%>>> elems = [10603,8091,11978,5523]
%>>> print mean(elems) , std (elems)
%9048.75 2467.00206475
%>>> elems = [ 2002,2678,8127,4200,10603, 1513,1975,6665,3341, 8091, 2393,3131,9013,4896,11978, 896,1225,4045,2629, 5523 ]
%>>> print mean(elems) , std (elems)
%4746.2 3196.00870775
%>>> nodes = [2580, 3388, 9394, 5200,12324, 2076, 2677, 7953, 4227, 9741 , 3029 , 3951,10352, 5974,13874 , 1349, 1806 , 5120, 3481, 6989]
%>>> print mean(nodes) , std (nodes)
%5774.25 3593.95211258
%>>> print min(nodes) , median (nodes) , max (nodes)
%1349 4673.5 13874
%>>> print min(elems) , median (elems) , max (elems)
%896 3693.0 11978


\textcolor{blue}{ % comments for review 1
The ROI for each canine, $\Omega \subset \mathbb{R}^3$ (FOV=
1.95x2.38cm $\pm$ .21x.19cm) 
was chosen large enough to encompass the
extent of the heating region but small enough to minimize bias of
the error metric \eqn{weightedL2Norm} where no  heating occurred.
Individual regions of the brain were not segmented due to the
localized nature of the heating study.
To ensure convergence, five FEM mesh resolutions were considered for
each ROI (min/median/max \# elements = 896/3693/11978, 
          min/median/max \# nodes    = 1349/4674/13874).
Within the ROI, the mesh size element diameter ranged between
[0.5mm,2mm]. As a reference, 1mm is on the order of the pixel size.
}


Geometric details of the applicator were closely modeled according
to the manufacturer design and details provided in
Figure~\ref{canineSetup}.  The active cooling of the applicator was
\textcolor{blue}{ % comments for review 1
modeled using Dirichlet boundary conditions to fix the temperature
}
along the axial length of the 1.5mm diameter catheter.  The
temperature of the catheter was studied at both ambient room
temperature, 21$^o$C, and body temperature 34.6$\pm$ 0.7 $^o$C.
The 400 $\mu$m core diameter active region of the fiber-optic was
modeled as $\Omega_{tip}$, a cylindrical 1.5mm diameter, 1cm axial
region, at a distance of 5mm from the catheter tip. Modeling at the
level of the 400 $\mu$m core diameter active tip is not expected to
produce significant differences in the computations.  The finite
element basis functions are continuous across the applicator tissue
interface.  

The optical photon source of the laser-tissue interaction was
rigorously derived from a diffusion theory approximation to the
light transport equation~\cite{Welch95}.  The resulting photon
source is referred to as the conformal discretization approximation
(CDA) in this manuscript.  The CDA model is similar to fully
analytic attempts to capture the cylindrical geometry of the
interstitial fiber-optic~\cite{Dickey04} but exploits the
underlying finite element discretization of the computational
domain to evenly distribute the laser power amongst the axial
length and finite diameter in the active tip region.  An analogous
discretization approach was implemented on structured
grids~\cite{arnfield1989optical} and results in multiple individual
optical diffusion approximation (ODA) along the centerline of the
applicator.  The CDA accounts for the finite diameter of the
applicator.  Within the diffusion theory assumptions, the resulting
conformal discretization approximation is expected to provide a
fluence source comparable to direct numerical solutions to the
light transport equation~\cite{Mohammed05,shafirstein2004}.

The diffusion model for light transport is built from the
assumption that light is scattered more than absorbed, $ \mu_a <<
\mu_s 
% \Rightarrow \mu_{eff} << \mu_t
$, where $\mu_a$ and $\mu_s $ give probability of absorption and
scattering of photons, respectively.  An applied volumetric power
density, $P^* \left[\frac{W}{m^3}\right]$, provides a photon flux
throughout the active region of the interstitial laser fiber,
$\Omega_{tip}$.  Huygens superposition principle~\cite{Dickey04} is
used to treat each position in the domain, $\hat{x}\in
\Omega_{tip}$, as an isotropic  differential point source of
irradiance, $dE$,
\[
dE(x,\hat{x})= 
             \frac{P^*(t)d\hat{x}}{4\pi 
    \| x - \hat{ x}\|^2}
             \exp\left(\mu_t
                         \| x - \hat{ x }\|\right) 
\qquad \qquad
\hat{\textbf{s}} = 
\frac{ x - \hat{ x }}{ \| x - \hat{ x}\| }
\qquad \qquad
x \in \Omega \backslash \Omega_{tip}
\]
The total attenuation is denoted $\mu_t  = {\mu_a} + {\mu_s}$.  The
propagation direction, $\hat{\textbf{s}}$, is the unit vector from
the primary source of unattenuated photons to the position $x$.
The light transport is assumed quasi-static and a stationary
diffusion model is used at each time point of interest to relate
the scattered fluence, $z$, to the known irradiance, $dE$, emitted
from point source $\hat{x} \in \Omega_{tip}$
%
% a unit vector along the line connecting the source (or element of the
% source) to the point x"
%
% "E_0(x,s) is the irradiance at point x in the absence of tissue and s
% indicates the direction of propagation of primary light. That is, s is
% a unit vector along the line connecting the source (or element of the
% source) to the point x"
%
% Use Welch...
% The radiance of the primary light written in terms of its irradiance 
% in equation (6.22). Substitutive this into eqn (6.16) leads to 
% the transport equation written in terms of the scattering light
% as the unknown variable with the unscattered light as the source,
% eqn (6.26). IE eqn (6.16) and (6.26) are EQUIVALENT given eqn (6.16)
% Under these assumptions 
%
% diffusion approximation introduced as a truncated series expansion of
% the radiance of scattered light eqn (6.27) and is used in the
% derivation of the flux conservation, eqn (6.32). The energy
% conservation, eqn (6.29), is EXACT.
%
% F in eqn (6.29) is  the net energy flux of the SCATTERED LIGHT 
% NOT THE SCATTERED PLUS IRRADIANT see eqn (6.28a) in Welch
%
% substituting  eqn (6.32b) into eqn (6.29) of welch book
\begin{equation} \label{odaLTE}
% \frac{1}{c(x)}
% \frac{\partial(z+E)}{\partial t}
 -{\mu_a} z 
 +{\mu_s} \; dE(x,\hat{x})
 = \nabla \cdot 
   \left( 
   - \frac{ \nabla z }{3\mu_{tr}} 
   + \frac{ {\mu_s} g }{3\mu_{tr}}
      \; dE(x,\hat{x}) \; \hat{\textbf{s}}
   \right)
\quad
  \mu_{tr}  = 
                 {\mu_a} + 
                    {\mu_s} (1-g)
\quad
x \in \Omega \backslash \Omega_{tip}
\end{equation}
The anisotropic factor is denoted $g$. 
A conformal discretization approximation (CDA)
of the integral of the differential irradiance over
the domain of the active tip, $\Omega_{tip}$,
is used to obtain an analytical expression
for the scattered light, $z$
\[
\begin{split}
E  & = \int_{\Omega_{tip}} dE(x,\hat{x}) d\hat{x}
     = \int_{\Omega_{tip}} 
             \frac{P^*(t)}{4\pi \| x - \hat{ x}\|^2}
             \exp\left(\mu_t
                         \| x - \hat{ x}\|\right) d\hat{x}
\\
  & \approx 
  \sum_{e\in \Omega_{tip}} E_e(x,t)
  =
  \sum_{e\in \Omega_{tip}} 
          \frac{\Delta V_e P^*(t)}{4\pi \| x - \hat{ x}_e\|^2}
          \exp\left(\mu_t 
                 \| x - \hat{ x}_e\|\right) 
\end{split} 
\]
where $\Delta V_e$ is the volume and $\hat{ x}_e$ is the
centroid of the elements within the FEM mesh of the active tip. 
Here, conformal refers to a discretization such that the finite
elements adhere to the boundary of the active tip, $\Omega_{tip}$.
Similarly for the flux,
%FIXME  FIXME  FIXME  FIXME  FIXME  FIXME  FIXME  FIXME  FIXME  FIXME
%Should ${\mu_t}$ be the average over the path length???
% this is already an approximation... if go through the trouble to 
% take \mu_t as the average, what will be gained... ie don't worry about
% it
%FIXME  FIXME  FIXME  FIXME  FIXME  FIXME  FIXME  FIXME  FIXME  FIXME
\[
\begin{split}
\int_{\Omega_{tip}} dE(x,\hat{x}) \; \hat{\textbf{s}} \; d\hat{x}
   &   = \int_{\Omega_{tip}} 
             \frac{P^*(t)}{4\pi \| x - \hat{ x}\|^2}
             \exp\left(\mu_t 
                         \| x - \hat{ x}\|\right) 
\frac{ x - \hat{ x} }{ \| x - \hat{ x}\| }
d\hat{x}
 \\
  & \approx 
    \sum_{e\in \Omega_{tip}} E_e(x,t) \hat{\textbf{s}}_e
  = \sum_{e\in \Omega_{tip}} 
          \frac{\Delta V_e P^*(t)}{4\pi \| x - \hat{ x}_e\|^2}
          \exp\left(\mu_t 
                         \| x - \hat{ x}_e\|\right) 
\frac{ x - \hat{ x}_e }{ \| x - \hat{ x}_e\| }
\end{split} 
\]
By linearity, each element in the discretization may be treated as
an uncoupled and independent source in the light diffusion
equation.
\[
 -{\mu_a} z_e 
 +{\mu_s} E_e
 = \nabla \cdot 
   \left( 
   - \frac{ \nabla z_e }{3\mu_{tr}} 
   + \frac{ {\mu_s} g }{3\mu_{tr}} 
       E_e 
      \frac{ x - \hat{ x}_e }{ \| x - \hat{ x}_e\| }
   \right) \qquad \forall e \in \Omega_{tip}
\]
The total fluence resulting from each element source, $(\phi_t)_e =
z_e + E_e$, may be obtained from the classical isotropic point
source solution~\cite{Welch95} as  the sum of the light scattered
from the element, $z_e$, and the element-wise primary source,
$E_e$.  The total emanating fluence, $\phi_t(x,t)$, is the
superposition of the element-wise solutions and reduces to a volume
weighted sum over the elements.
\begin{equation} \label{wfsLaserFluence}
\begin{split}
   \phi_t(x,t) & = \sum_{e \in \Omega_{tip}}
 {P^*(t)} \Delta  V_e  \left(
   \frac{3\mu_{tr} \exp(-\mu_{eff} \| x -{ x_e}\|) }
      {4\pi \| x-{ x_e}\|}
 - 
      \frac{ 
             \exp(-\mu_t \| x -{ x_e}\|) }
           {2\pi \| x-{ x_e}\|^2}
   \right)
\\
& \approx
    \sum_{e \in \Omega_{tip}}
 {P^*(t)}  \Delta V_e 
   \frac{3\mu_{tr} \exp(-\mu_{eff} \| x -{ x_e}\|) }
      {4\pi \| x-{ x_e}\|}
\qquad
 \mu_{eff}  = 
           \sqrt{ 3 {\mu_a} \mu_{tr} }
\\
& 
\hspace{3in}
 x \in \Omega \backslash \Omega_{tip}
\end{split}
\end{equation}

A single point source ODA of the photon source at the centroid of
the active tip, $ x_0$, is used as a control and comparison against
previous work~\cite{fuentesetal09}.
\begin{equation} \label{odaLaserFluence}
   \phi_t(x,t)  = 
   P(t) 
   \frac{3\mu_{tr} \exp(-\mu_{eff} \| x -{ x_0}\|) }
      {4\pi \| x-{ x_0}\|}
\qquad  x \in \Omega \backslash \Omega_{tip}
\end{equation}

A linear Pennes model was used to simulate the bioheat
transfer~\cite{Pennes48}.  The fluence, $\phi_t$, provides the
thermal source in the bioheat equation and couples the diffusion
theory model of light transport in tissue to Pennes model.  The
laser powers and exposure times for each canine shown in
Figure~\ref{data980CanineSummary} were modeled using piecewise
continuous step functions of the wattage.  The model considers the
optical and thermal properties of the tissue as well as a highly
simplified model of the micro vasculature heat sink consisting of a
linear temperature difference between the heated region and the
arterial temperature, $u_a$. The Pennes bioheat equation has been
shown to accurately model heating in areas absent of major
vasculature~\cite{arkin1994}. 
\[ \begin{split}
 \rho  c_p \frac{\partial u}{\partial t}
 -\nabla \cdot ( {k} \nabla u) 
 +{\omega} c_{blood} &(u - u_a )
 = {\mu_a} \phi_t  \qquad \text{in } \Omega
\\
   -  k  \nabla u \cdot \textbf{n} = 0
           \qquad \text{on } \partial \Omega
  &\qquad \qquad 
   u  = u_{probe} \qquad \text{in } \Omega_{probe}
\end{split} 
\]
The initial temperature field, $u( x,0)=u^0=$ 34.6$\pm$ 0.7 $^o$C,
is taken as the measured baseline body temperature.  The density of
the continuum is denoted $\rho$ and the specific heat of blood is
denoted $c_{b} \left[\frac{J}{kg \cdot K}\right]$.  The boundary of
the mesh template is far enough away from the heating region such
that zero heat flux may be considered as the Neumann boundary
condition on $\partial \Omega$.  The scalar-valued coefficient of
thermal conductivity and perfusion are denoted $k$ and $\omega$.
Within the region of the mesh that represents the applicator,
$\Omega_{probe}$, a Dirichlet boundary condition is imposed such
that the cooling water of the applicator is held at body
temperature, $u_{probe}=u^0$, or ambient temperature,
$u_{probe}=$21$^o$C, throughout the procedure.  The temperature
evolution predicted by the bioheat equation was computed using the
finite element method with linear polynomials and a Crank-Nicholson
time stepping scheme.

%%%%%%%%%%%%%%%%%%%%%%%%%%%%%%%%%%%%%%%%%%%%%%
\subsubsection{Constitutive Data}
%%%%%%%%%%%%%%%%%%%%%%%%%%%%%%%%%%%%%%%%%%%%%%

The bio-thermal and optical parameters were obtained from
literature~\cite{Handbook05,Welch95,duck1990} and were modeled as
homogeneous throughout the delivery region of interest,
Table~\ref{modeldata}.  An average value of the range of
scattering, $\mu_s$, and absorption, $\mu_a$, values quoted in the
literature was used.  A range of perfusion values
$\omega$=0.0,3.0,6.0,12.0$[\frac{kg}{s m^3}]$ were simulated to
capture physically realistic values of brain tissue perfusion found
in literature and as a first order study of temperature dependent
perfusion.  Empirical models of the temperature dependence of the
constitutive data~\cite{pegau1997absorption,Duggan00,Handbook05},
$k(u)$, $\omega(u)$, $\mu_a(u)$, will be considered in future
validation studies.  Statistics were collected comparing the
thermal imaging data for each animal against each permutation of
laser model, cooling model, and perfusion value, a total of 64
simulations at each mesh resolution considered. 

\begin{table}[h]
\caption{Constitutive Data~\cite{Handbook05,Welch95,duck1990}}\label{modeldata}
\centering 
\begin{tabular}{|c|c|c|c|c|c|c|c|} \hline 
$k $ $ \frac{W}{ m \cdot K}$ & $\omega$ $\frac{kg}{m^3 s}$ &  $g$  &  $\mu_s$ $\frac{1}{cm}$  &  $\mu_a$  $\frac{1}{cm}$   &  $\rho$ $\frac{kg}{m^3}$ &   $c_{blood}$ $ \frac{J}{kg \cdot K}$ &  $c_p$ $\frac{J}{kg \cdot K}$ \\ \hline
          0.527              &     0.0,3.0,6.0,12.0        & 0.862 &        47.0              &       0.45                 &  1045                        &            3840                      &                  3600          \\ \hline
\end{tabular}
\end{table}

%===================================================================
\subsection{Error Metric and Thermal Image Noise}
%===================================================================

The temperature dependent relaxation properties of tissue are
expected to adversely affect the MR temperature signal and cause
artifacts.  Time varying and spatially dependent maps of the
temperature image noise, $\sigma(x,t)$, are used to pointwise
normalize unrealistic discrepancies between the measured
temperature data, $u^{MRTI}$, and the model predicted temperature,
$u$. The resulting metric of comparison, $\epsilon(t)$, is a
weighted space $L_2$ norm within the time interval of interest,
$[0,T]$, and normalized by the volume of the ROI, 
$\Omega \subset \mathbb{R}^3$
\begin{equation}\label{weightedL2Norm}
\epsilon(t) = 
\left( 
\frac{ \|u - u^{MRTI}\|^2_\sigma }{ \int_\Omega \; dx }
\right)^{1/2}
 = 
\left( 
  \frac{ 
    \int_\Omega \left(
                     \frac{u(x,t) - u^{MRTI}(x,t)}{\sigma(x,t)}
               \right)^2 \; dx 
       }{ \int_\Omega \; dx }
\right)^{1/2}
   \qquad t \in [0,T]
\end{equation}
The SNR is assumed sufficiently large (SNR $>$ 10) such that the
noise in the real and imaginary images may be approximated as
Gaussian and the temperature image noise may be obtained from the
measured SNR of the MR signal.
\[
\sigma(x,t) = \max_{\tau \in [0,t] \subseteq [0,T]}
              \frac{ 1 } { 2 \pi \alpha \cdot \gamma B_0 \cdot \text{TE}}
              \frac{ \sqrt{2} }{ \text{SNR}(x,\tau) }
\]
The current time maximum, $\max_{\tau \in [0,t]}$, is needed to
normalize unrecoverable phase changes following T1 related signal
loss due to large temperature changes.  The pointwise SNR was
obtained as the ratio of corresponding magnitude image intensity
$I(x,t)$ to the measured system noise, $\sigma_{system}$. 
\[
\sigma_{system} = \frac{<\delta I>_{ROI}}{0.655 \sqrt{2}}
\]
A difference method~\cite{Reeder07} was used to compute the system
noise in the corresponding magnitude images.  The difference image
standard deviation in the small volume of interest, $<\delta
I>_{ROI}$, in air is measured in initial magnitude images.  A
factor of $0.655$ is needed to approximate the low SNR Raleigh
distributed signal in air as a Gaussian
distribution~\cite{Haacke99}. The $\sqrt{2}$ arises from the
difference method.

\begin{center}
 \textit{Possible location of Figure~\ref{data980CanineSummary}}
\end{center}




\paragraph{Acknowledgments}
%If you'd like to thank anyone, place your comments here
%and remove the percent signs.
The research in this paper was supported in part through the NIH
5T32CA119930-03 funding mechanism.  Canine data was obtained from
BioTex, Inc., under grants R43-CA79282, R44-CA79282, R43-AG19276.
The authors would also like to thank the
ITK~\cite{ITKSoftwareGuideSecondEdition}, Paraview~\cite{Paraview},
PETSc~\cite{petsc-manual}, libMesh~\cite{libMesh}, and
CUBIT~\cite{cubit} communities for providing truly enabling
software for scientific computation and visualization.  The initial
3D canine model in Figure~\ref{canineSetup} was provided by AddZero
and obtained from the official
Blender~\cite{roosendaal2000official} model repository.  Parameter
studies were performed using allocations  at the Texas Advanced
Computing Center. 

\end{document}
% end of file template.tex



